\documentclass[12pt]{amsart}
% this is here to force arXiv to produce a nice output
\pdfoutput=1

\usepackage{mathtools}
\usepackage{amsmath}
\usepackage{amsthm}
\usepackage{amssymb}
\usepackage{amsbsy}
\usepackage{amstext}
\usepackage{amsopn}
\usepackage{enumerate}
\usepackage{xcolor}
\usepackage{graphicx}
\usepackage{microtype}
\usepackage{verbatim}
\usepackage[margin=1in,marginparwidth=0.8in, marginparsep=0.1in]{geometry}
\renewcommand{\baselinestretch}{1.2} % changes page formatting
\usepackage[bookmarks=true, bookmarksopen=true, bookmarksdepth=3,bookmarksopenlevel=2, colorlinks=true, linkcolor=blue, citecolor=blue, filecolor=blue, menucolor=blue, urlcolor=blue]{hyperref}
\usepackage{newtxtext} % changes font appearance, replaces times
\usepackage{stmaryrd}
\usepackage{accents}
\usepackage{bbm}

% For setting off new terms
\newcommand{\newword}[1]{\textbf{\emph{#1}}}

% shorthands
\DeclareMathOperator{\corank}{corank}
\DeclareMathOperator{\Ext}{Ext}
\DeclareMathOperator{\Hom}{Hom}
\DeclareMathOperator{\cl}{cl}
\DeclareMathOperator{\Cone}{Cone}
\newcommand{\Bdp}{\widetilde{B}_{dp}}
\newcommand{\bfa}{\mathbf{a}}
\newcommand{\bfc}{\mathbf{c}}
\newcommand{\bfg}{\mathbf{g}}
\newcommand{\bfd}{\mathbf{d}}
\newcommand{\bfi}{\mathbf{i}}
\newcommand{\Bpr}{\widetilde{B}_{pr}}
\newcommand{\cA}{\mathcal{A}}
\newcommand{\cC}{\mathcal{C}}
\newcommand{\cE}{\mathcal{E}}
\newcommand{\cF}{\mathcal{F}}
\newcommand{\cD}{\mathcal{D}}
\newcommand{\cN}{\mathcal{N}}
\newcommand{\cP}{\mathcal{P}}
\newcommand{\cQ}{\mathcal{Q}}
\newcommand{\cR}{\mathcal{R}}
\newcommand{\cv}{\alpha}
\newcommand{\dashname}[1]{\stackrel{#1}{\begin{picture}(22,3)\put(0,2.5){\line(1,0){22}}\end{picture}}}
\newcommand{\dol}[1]{\overline{\overline{#1}}}
\newcommand{\grep}{\gv}
\newcommand{\Gr}{\mathrm{Gr}}
\newcommand{\gv}{\omega}
\newcommand{\into}{\hookrightarrow}
\newcommand{\kk}{\Bbbk}
\newcommand{\KQ}{K_0(Q)}
\newcommand{\loopvar}{z}
\newcommand{\NN}{\mathbb{N}}
\newcommand{\ol}[1]{\overline{#1}}
\newcommand{\onto}{\twoheadrightarrow}
\newcommand{\Qdp}{\widetilde{Q}_{dp}}
\newcommand{\Qrep}{M}
\newcommand{\rep}{\operatorname{rep}}
\newcommand{\TT}{\mathbb{T}}
\newcommand{\Zidx}{\ell}
\newcommand{\ZZ}{\mathbb{Z}}
\newcommand{\RR}{\mathbb{R}}
\newcommand{\CC}{\mathbb{C}}
\renewcommand{\mod}[1]{\langle {#1} \rangle}
\newcommand{\congto}{\xrightarrow{\sim}}
\newcommand{\wh}{\widehat}

% Commands for marginal notes below
\usepackage[draft]{say}
\newcommand{\sayS}[1]{\say[S]{#1}}


%%%%%%%%%%%%%%%%%%%%%%%%%%%%%%%%%%%%%%%%%%%%%%%%%%
% Macro double Bruhat cells and their exponents
%%%%%%%%%%%%%%%%%%%%%%%%%%%%%%%%%%%%%%%%%%%%%%%%%%
\usepackage{xparse}
% Short version
\NewDocumentCommand{\dbc}{ m O{#1} }{G^{#1}}
% Long version
%\NewDocumentCommand{\dbc}{ m O{{#1}^{-1}} }{G^{#1,#2}}

\newcommand{\muoc}{s_1cs_1}
\newcommand{\muoci}{s_1c^{-1}s_1}
\newcommand{\munc}{s_ncs_n}
\newcommand{\munci}{s_nc^{-1}s_n}

%\newcommand{\muoc}{c_1}
%\newcommand{\muoci}{c_1^{-1}}
%\newcommand{\muoc}{\mu_1\hspace{-1 pt}(c)}
%\newcommand{\muoci}{\mu_1\hspace{-1 pt}(c)^{-1}}
%\newcommand{\munc}{c_n}
%\newcommand{\munci}{c_n^{-1}}
%\newcommand{\munc}{\mu_n\hspace{-1 pt}(c)}
%\newcommand{\munci}{\mu_n\hspace{-1 pt}(c)^{-1}}
%%%%%%%%%%%%%%%%%%%%%%%%%%%%%%%%%%%%%%%%%%%%%%%%%%
% done
%%%%%%%%%%%%%%%%%%%%%%%%%%%%%%%%%%%%%%%%%%%%%%%%%%

% ambients and numbering
\newtheorem{theorem}{Theorem}[section]
\newtheorem{conjecture}[theorem]{Conjecture}
\newtheorem{corollary}[theorem]{Corollary}
\newtheorem{definition}[theorem]{Definition}
\newtheorem{lemma}[theorem]{Lemma}
\newtheorem{proposition}[theorem]{Proposition}
\theoremstyle{definition}
\newtheorem{example}[theorem]{Example}
\newtheorem{remark}[theorem]{Remark}
\numberwithin{equation}{section}
\numberwithin{figure}{section}

\begin{document}
\title{Comparison with Fan's construction}

\author[Dylan Rupel]{Dylan Rupel}
\address[Dylan Rupel]{University of Notre Dame, Department of Mathematics, Notre Dame, IN 46556, USA}
\email{drupel@nd.edu}

\author[Salvatore Stella]{Salvatore Stella}
\address[Salvatore Stella]{University of Haifa, Department of Mathematics, Haifa, 31905, Israel}
\email{stella@math.haifa.ac.il}

\author[Harold Williams]{Harold Williams}
\address[Harold Williams]{University of California, Davis, Department of Mathematics, Davis, CA 95616, USA}
\email{hwilliams@math.ucdavis.edu}

%\keywords{Cluster algebras, Kac-Moody groups}
%\subjclass[2010]{13F60, 20G44}
%%%%%%%%%%%%%%%%%%%%%%%%%%%%%%%%%%%%%%%%%%%%%%%%%%%
%
%\setcounter{tocdepth}{1}
%
%\maketitle
%
%\tableofcontents
%
%\vspace{1cm}
%
%%%%%%%%%%%%%%%%%%%%%%%%%%%%%%%%%%%%%%%%%%%%%%%%%%%

We want to show that our continuous family of generalized minors from \cite[Theorem 4.6]{RSW19} coincide with the family of bases constructed in \cite[Theorem 1.2.1]{Qin19}.

This boils down to show, for any $n>1$, that
\begin{enumerate}
  \item 
    the set of $\bfg$-vectors that are dominated by $n\omega^\circ$ with respect to any seed is 
    \[
      \left\{(n-2r)\omega^\circ \right\}_{r\in\left[1,\left\lfloor\frac{1}{2}\right\rfloor\right]}
    \]

  \item
    we can choose a specific basis and rewrite the last formula in \cite[Proposition 4.4]{RSW19} as in \cite[Theorem 1.2.1]{Qin19} with coefficients being functions of the point $\bfa\in(\kk^\times)^n$. 
\end{enumerate}

For a partition $\lambda \vdash n$ let $e_\lambda$, $s_\lambda$, and $m_\lambda$ denote the elementary, Shur, and monomial symmetric functions associated to the partition $\lambda$ evaluated at $\bfa$.
Recall that $m_{1^{(n)}} = s_{1^{(n)}} = e_n$.

\begin{lemma}
  For every $r\in\left[1,\left\lfloor\frac{1}{2}\right\rfloor\right]$ we have $m_{1^{(n)}} S_{\bfa,r} = m_{2^{(r)},1^{(n-2r)}}$.
\end{lemma}
\begin{proof}
  This is immediate from the definition.
\end{proof}

Let $F_{n\omega^\circ}^\Delta$, $F_{n\omega^\circ}^{gr}$, $F_{n\omega^\circ}^{tr}$, and $F_{n\omega^\circ}^{ge}$ be the $F$-polynomials of the generalized minor, greedy basis element, triangular basis element, and generic basis element with $\bfg$-vector $n\omega^\circ$.
We think of $F$-polynomials as polynomials in the variables $u_1$ and $u_2$.

I did not try yet to prove (1) but it should be straightforward.
It suffices to establish (2) for the $F$-polynomials.

\begin{proposition}
  \[
    m_{1^{(n)}} F_{n\omega^\circ}^\Delta
    =
    \sum_{r=0}^{\lfloor \frac{n}{2} \rfloor}
    m_{2^{(k)},1^{(n-2r)}}
    u_1^r u_2^r
    F_{(n-2r)\omega^\circ}^{ge}
  \]
\end{proposition}
\begin{proof}
  From \cite[Proposition 4.4]{RSW19} we get the following expression for $F_{n\omega^\circ}^\Delta$:
  \[
     F_{n\omega^\circ}^\Delta 
     =
     \sum_{0 \le k \le \ell \le n}
     \sum_{r=0}^\ell
     {\ell -r \choose k}
     {n-2r \choose \ell-r}
     S_{\bfa,r}
     u_1^\ell u_2^{\ell-k}.
   \]
  From the second binomial coefficient we deduce that $n-2r$ is positive so the second sum runs up to $\lfloor \frac{n}{2} \rfloor$.
  We compute
  \begin{align*}
    m_{1^{(n)}}  F_{n\omega^\circ}^\Delta 
    &
    =
    \sum_{0 \le k \le \ell \le n}
    \sum_{r=0}^{\lfloor \frac{n}{2} \rfloor}
    {\ell -r \choose k}
    {n-2r \choose \ell-r}
    m_{2^{(r)},1^{(n-2r)}}
    u_1^\ell u_2^{\ell-k}
    \\
    &=
    \sum_{r=0}^{\lfloor \frac{n}{2} \rfloor}
    m_{2^{(r)},1^{(n-2r)}}
    u_1^r u_2^r
    \sum_{0 \le k \le \ell \le n}
    {\ell -r \choose k}
    {n-2r \choose \ell-r}
    u_1^{\ell-r} u_2^{\ell-r-k}
    \\
    &=
    \sum_{r=0}^{\lfloor \frac{n}{2} \rfloor}
    m_{2^{(r)},1^{(n-2r)}}
    u_1^r u_2^r
    F_{(n-2r)\omega^\circ}^{ge}.
  \end{align*}
  The last identity follows immediately by comparing with the formula for generic basis elements in \cite{RSW19}.
  \sayS{Thank you Dylan for fixing my stupidity.}
\end{proof}

We do not need the following two formulas but I think they are still neat.
\begin{conjecture}
  \begin{align*} 
    m_{1^{(n)}} F_{n\omega^\circ}^\Delta
    &=
    \sum_{r=0}^{\lfloor \frac{n}{2} \rfloor}
    s_{2^{(k)},1^{(n-2r)}}
    u_1^r u_2^r
    F_{(n-2r)\omega^\circ}^{tr}
    \\&=
    \sum_{r=0}^{\lfloor \frac{n}{2} \rfloor}
    e_{r,n-r}
    u_1^r u_2^r
    F_{(n-2r)\omega^\circ}^{gr}
  \end{align*}
\end{conjecture}

\bibliographystyle{amsalpha}
\bibliography{bibliography}

\end{document}
