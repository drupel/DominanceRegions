\documentclass{amsart}

\title{Response to Dominance\_Regions\_Review.pdf}

\begin{document}
  
  \maketitle

  We have addressed each of the concerns in the bullet points below.  We welcome any further comments or concerns.
  \begin{enumerate}
    \item The first two paragraphs seem to address this.
      An additional comment was added as well.
    \item Fixed.
    \item We haven't investigated the phenomenon thoroughly enough to feel confident including a more precise comment in the paper, but it seems that only points in the dominance region near the origin are appearing.
    \item We have adjusted the exposition to remove the need for this notation.
    \item The notion is standard in the theory of cluster algebras.
      The reason for this terminology lies outside the scope of the paper, but for the benefit of the referee we provide as explanation here.
      Each greedy element (or, more generally, theta basis element) has an associated denominator vector.
      In rank two, these vectors are imaginary roots of the associated root system precisely when the $\mathbf{g}$-vector of the greedy element lies in the imaginary cone.
    \item We do not know why this should occur, an additional comment has been added to Remark 1.5.
    \item Caption and detailed description added.
    \item Caption added.
    \item Figure numbers and references added.
  \end{enumerate}

\end{document}
